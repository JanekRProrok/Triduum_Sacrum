\documentclass[10pt,oneside,final,notitlepage,a4paper,wide]{mwart}
\usepackage[utf8]{inputenc} 
\usepackage{polski} 
\usepackage{graphicx}
\usepackage{setspace}

\onehalfspacing
\setlength{\parindent}{1cm}

\usepackage[]{hyperref}
\usepackage{hyperref,xcolor}% http://ctan.org/pkg/{hyperref,xcolor}
\usepackage{makeidx}

\definecolor{orangelink}{rgb}{0.7,0.18,0.1}

\hypersetup{
    pdftitle={Missa Vespertina in Cena Domini},
    pdfdisplaydoctitle=true,
    pdfauthor={pitrk, JanekR_Prorok},
    pdfsubject={Source: https://github.com/JanekRProrok/Triduum_Sacrum},
    pdfcreator={Texmaker, MiKTeX},
    pdfproducer={},
    pdfinfo={},
    pdfkeywords={},
    bookmarks=true,
    bookmarksnumbered=true,
    bookmarksopen=true,
    bookmarksopenlevel=1,
    pdfpagelabels=true,
    pdfpagemode=UseOutlines,
    unicode=true,
    pdftoolbar=true,
    pdfmenubar=true,
    pdffitwindow=false,
    pdfstartview=Fit,
    pdfnewwindow=true,
    colorlinks=true,
    linkcolor=orangelink,
    citecolor=orangelink,
    filecolor=orangelink,
    urlcolor=orangelink,
    pdfpagelayout=TwoPageRight,
}
% http://www.tug.org/applications/hyperref/manual.html#x1-120003.8

%------------------------------------------------------------------------------%
%------------------------------------------------------------------------------%

	% Wypełnienie ... \dotline{długość+0,1cm}.
\def\dotfill#1{\cleaders\hbox to #1{.}\hfill}
\newcommand\dotline[2][0,1cm]{\leavevmode\hbox to #2{\dotfill{#1}\hfil}}	
	
%------------------------------------------------------------------------------%
	
\begin{document}
% \sloppy - Wymuszenie nieprofesjonalnego trzymania się w marginesach (kosztem dziwnie rozciągniętych odstępów).
%
\begin{center}
	\LARGE{\textbf{Missa Vespertina in Cena Domini}}\\ \smallskip
	\small{Msza wieczorna w Wielki Czwartek\\ \smallskip Stacja u Św. Jana na Lateranie}
\end{center} \vspace{1cm}

	Kościół rozpoczyna dziś celebrację Triduum Paschalnego, będącego centrum całego roku liturgicznego. Podczas tych trzech świętych dni uobecniane są w liturgii najważniejsze tajemnice z życia Jezusa Chrystusa -- Jego śmierć i zmartwychwstanie.

W godzinach przedpołudniowych w kościele katedralnym, biskup ordynariusz odprawił Mszę Krzyżma Świętego, na której zostały poświęcone oleje święte -- olej chorych, używany przy namaszczeniu chorych, Krzyżmo święte -- materia sakramentu Bierzmowania, używane również między innymi przy konsekracji kościołów i ołtarzy, konsekracji biskupów i święceniach kapłańskich, oraz olej katechumenów, używany przy chrzcie. Olej ma liczne i różne zastosowania. Służy jako pokarm, jako lekarstwo, a także jako źródło światła. Podobnie jego symbolika w liturgii jest bogata. Jest symbolem siły, ukojenia i oświecenia.

Wielki Czwartek jest przede wszystkim rocznicą ustanowienia sakramentu Eucharystii i Kapłaństwa. Eucharystia ustanowiona w czasie Ostatniej Wieczerzy jest owocem i uobecnieniem ofiary krzyżowej. Chleb i wino, które Pan Jezus wybrał na postacie Najświętszego Sakramentu, symbolizują zjednoczenie. Wiele ziaren łączy się w jednym chlebie, sok wielu jagód tworzy wino.

Dziś podczas śpiewu Gloria uderzą dzwony, które następnie wraz z organami zamilkną aż do śpiewu tego samego hymnu podczas liturgii Wigilii Paschalnej.

W trakcie Mszy Wieczerzy Pańskiej, kapłan konsekruje komunikanty przeznaczone również na obrzędy komunii w dniu jutrzejszym, a także hostię przeznaczoną do wystawienia w Bożym Grobie, gdyż w Wielki Piątek Eucharystia nie jest sprawowana.

Po modlitwie po komunii, Najświętszy Sakrament w cyborium zostanie przeniesiony do ołtarza wystawienia. Za pobożne odśpiewanie hymnu \emph{Przed tak wielkim Sakramentem} można dzisiaj uzyskać, pod zwykłymi warunkami, odpust zupełny. Adoracja Najświętszego Sakramentu będzie trwała do godziny \dotline{1.6cm}.

Na zakończenie, nastąpi obnażenie ołtarza -- symbolu samego Chrystusa Pana. Obnażenie ołtarza przypomina, że w czasie męki odarto Zbawiciela z szat, usiłując zbezcześcić Jego ludzką godność.

W dniu jutrzejszym obowiązuje zachowanie wstrzemięźliwości od pokarmów mięsnych i post ścisły, pozwalający na jeden posiłek do syta i dwa skromne posiłki w ciągu dnia. Prawem o wstrzemięźliwość są związane osoby, które ukończyły czternasty rok życia, prawem zaś o poście są związane wszystkie osoby pełnoletnie, aż do rozpoczęcia sześćdziesiątego roku życia.

Piątkowa liturgia ku czci Męki i Śmierci Pańskiej w naszej kaplicy rozpocznie się o godzinie \dotline{1.6cm}.
\end{document}
