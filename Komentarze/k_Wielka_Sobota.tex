\documentclass[10pt,oneside,final,notitlepage,a4paper,wide]{mwart} 
\usepackage[utf8]{inputenc} 
\usepackage{polski} 
\usepackage{graphicx}
\usepackage{setspace}

\onehalfspacing
\setlength{\parindent}{1cm} 

\usepackage[]{hyperref}
\usepackage{hyperref,xcolor}% http://ctan.org/pkg/{hyperref,xcolor}
\usepackage{makeidx}

\definecolor{orangelink}{rgb}{0.7,0.18,0.1} 

\hypersetup{
    pdftitle={Dominica Paschæ in Resurrectione Domini - Ad Vigiliam Paschalem in Nocte Santa},
    pdfdisplaydoctitle=true,
    pdfauthor={pitrk, JanekR_Prorok},
    pdfsubject={Source: https://github.com/JanekRProrok/Triduum_Sacrum},
    pdfcreator={Texmaker, MiKTeX},
    pdfproducer={},
    pdfinfo={},
    pdfkeywords={},
    bookmarks=true,
    bookmarksnumbered=true,
    bookmarksopen=true,
    bookmarksopenlevel=1,
    pdfpagelabels=true,
    pdfpagemode=UseOutlines,
    unicode=true,
    pdftoolbar=true,
    pdfmenubar=true,
    pdffitwindow=false,
    pdfstartview=Fit,
    pdfnewwindow=true,
    colorlinks=true,
    linkcolor=orangelink,
    citecolor=orangelink,
    filecolor=orangelink,
    urlcolor=orangelink,
    pdfpagelayout=TwoPageRight,
}
% http://www.tug.org/applications/hyperref/manual.html#x1-120003.8

%------------------------------------------------------------------------------%
%------------------------------------------------------------------------------%

	% Wypełnienie ... \dotline{długość+0,1cm}.
\def\dotfill#1{\cleaders\hbox to #1{.}\hfill}
\newcommand\dotline[2][0,1cm]{\leavevmode\hbox to #2{\dotfill{#1}\hfil}}	
	
%------------------------------------------------------------------------------%
	
\begin{document}
% \sloppy - Wymuszenie nieprofesjonalnego trzymania się w marginesach (kosztem dziwnie rozciągniętych odstępów).
%
\begin{center}
	\LARGE{\textbf{Dominica Paschæ in Resurrectione Domini\\Ad Vigiliam Paschalem in Nocte Santa}}\\ \smallskip
	\small{Niedziela Zmartwychwstania Pańskiego\\Wigilia Paschalna w Wielką Noc\\ \smallskip Stacja u Św. Jana na Lateranie}
\end{center} \vspace{1cm}

	\textbf{Przed rozpoczęciem liturgii} Niedziela Zmartwychwstania jest głównym świętem roku kościelnego. Wszystkie inne niedzielę są wspomnieniem tej niedzieli nad niedzielami.
\par Obchód Zmartwychwstania Pańskiego obejmuje: Wigilię Wielkanocną z uroczystą nocną Mszą Zmartwychwstania, procesję rezurekcyjną i Mszę w dzień.
\par Wigilia to pierwotna nazwa pierwszej części publicznej modlitwy Kościoła, którą dziś nazywamy Jutrznią. Chociaż Wigilia Wielkanocna zaczyna się w ostatnich godzinach Wielkiej Soboty, treścią należy do liturgii Niedzieli Zmartwychwstania.
\par Nocne obchody składać się będą z czterech części: Liturgii Światła, Liturgii Słowa, Liturgii Chrzcielnej i Liturgii Eucharystycznej.
\par Za chwilę kapłan przewodniczący dzisiejszej uroczystości wraz z całą asystą uda się przed wejście do kaplicy, gdzie dokona poświęcenia ognia i paschału.
\par Aby nam dopomóc w poznaniu rzeczy niewidzialnych, Kościół zwykle posługuje się znakami materialnymi. Liturgia Wielkiej Nocy zmierza do tego, aby przez kontrast ciemności i światła, ognia i wody uzmysłowić nam prawdę, że ze śmierci Chrystusa wytrysnęło życie.
\par W liturgii noc jest symbolem mocy szatana i śmierci. Chrystus zmartwychwstając pokonał szatana i śmierć, odnowił nasze życie nadprzyrodzone i dał nam rękojmię zmartwychwstania naszych ciał. Symbolem Chrystusa zmartwychwstałego jest wielkanocna świeca: Paschał. Wśród uroczystych obrzędów poświęca się ogień, którym paschał ma być zapalony, a później samą świecę.
\par Następnie Paschał zostanie w uroczystej procesji wniesiony do ciemnej kaplicy. Procesja ze światłem to obrazowe przedstawienie Zmartwychwstania, wywodzące się z Jerozolimy. Duchowieństwo tamtejsze zamykało się w Grobie Pańskim i wychodziło z niego z symbolicznym światłem, podobnie jak Chrystus wyszedł z grobu w blasku swej chwały.
\par Podczas procesji trzykrotnie usłyszymy śpiew \emph{,,Światło Chrystusa!''}, na który odpowiemy \emph{,,Bogu niech będą dzięki''}. Po drugim śpiewie zapalimy nasze świece od paschału, które będą symbolizowały przyjęcie zmartwychwstałego Chrystusa do naszego życia. Prosimy o odpalanie świec od ministrantów i przekazanie sobie światła.
\par Po procesji, w postawie stojącej z zapalonymi świecami, wysłuchamy radosnego śpiewu \emph{Exultet}. \bigskip

	\textbf{Liturgia Słowa} Usiądźmy. Prosimy o zgaszenie świec.
\par Stół Słowa Bożego jest tej nocy szczególnie bogato zastawiony. Usłyszymy za chwilę \dotline{0.6cm} czyta\dotline{1.1cm} ze Starego Testamentu. Po nich odśpiewamy hymn \emph{,,Gloria''} podczas którego znów zabiją dzwony i zagrają organy. W pierwotnym Kościele, hymn ten był śpiewany tylko na Wielkanoc. Następnie po odczytaniu fragmentu Listu świętego Pawła zaśpiewamy uroczyste \emph{,,Alleluia''}.
\par Dzisiejsze czytania ukazują stopniowy rozwój dzieła Odkupienia i zapowiadają odrodzenie w Chrystusie. \newpage % \bigskip

	\textbf{Liturgia Chrzcielna} Przechodzimy teraz do kolejnej części dzisiejszej uroczystości, do Liturgii Chrzcielnej. Od najdawniejszych czasów Kościół łączył tę noc Paschalną z Sakramentem Chrztu. W sakramencie tym otrzymaliśmy najpełniejszy udział w tajemnicy śmierci i zmartwychwstania Chrystusa.
\par W najważniejszych obrzędach, Kościół wzywa zawsze orędownictwa Wszystkich Świętych. Czyni to również przystępując do poświęcenia wody chrzcielnej. Po poświęceniu wody, wspominając własny chrzest, odnowimy nasze przyrzeczenia chrzcielne, za co dzisiaj możemy uzyskać, pod zwykłymi warunkami, odpust zupełny.
\par Powstańmy i zapalmy świece. \bigskip

	\textbf{Liturgia Eucharystyczna} Odnowiwszy w sobie łaskę Chrztu, który nas wszczepił w misterium śmierci i Zmartwychwstania Pańskiego, składamy ofiarę Mszy Świętej. W tej Ofierze w sakramentalny sposób stanie się obecne misterium paschalne, to jest tajemnicze przejście Chrystusa przez śmierć do niego życia.
\end{document}
