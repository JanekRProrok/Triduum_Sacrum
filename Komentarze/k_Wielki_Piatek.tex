\documentclass[10pt,oneside,final,notitlepage,a4paper,wide]{mwart}
\usepackage[utf8]{inputenc} 
\usepackage{polski} 
\usepackage{graphicx}
\usepackage{setspace}

\onehalfspacing
\setlength{\parindent}{1cm}

\usepackage[]{hyperref}
\usepackage{hyperref,xcolor}% http://ctan.org/pkg/{hyperref,xcolor}
\usepackage{makeidx}

\definecolor{orangelink}{rgb}{0.7,0.18,0.1}

\hypersetup{
    pdftitle={Feria sexta in Passione et Morte Domini},
    pdfdisplaydoctitle=true,
    pdfauthor={pitrk, JanekR_Prorok},
    pdfsubject={Source: https://github.com/JanekRProrok/Triduum_Sacrum},
    pdfcreator={Texmaker, MiKTeX},
    pdfproducer={},
    pdfinfo={},
    pdfkeywords={},
    bookmarks=true,
    bookmarksnumbered=true,
    bookmarksopen=true,
    bookmarksopenlevel=1,
    pdfpagelabels=true,
    pdfpagemode=UseOutlines,
    unicode=true,
    pdftoolbar=true,
    pdfmenubar=true,
    pdffitwindow=false,
    pdfstartview=Fit,
    pdfnewwindow=true,
    colorlinks=true,
    linkcolor=orangelink,
    citecolor=orangelink,
    filecolor=orangelink,
    urlcolor=orangelink,
    pdfpagelayout=OneColumn,
}
% http://www.tug.org/applications/hyperref/manual.html#x1-120003.8

%------------------------------------------------------------------------------%
%------------------------------------------------------------------------------%

	% Wypełnienie ... \dotline{długość+0,1cm}.
\def\dotfill#1{\cleaders\hbox to #1{.}\hfill}
\newcommand\dotline[2][0,1cm]{\leavevmode\hbox to #2{\dotfill{#1}\hfil}}	
	
%------------------------------------------------------------------------------%
	
\begin{document}
% \sloppy - Wymuszenie nieprofesjonalnego trzymania się w marginesach (kosztem dziwnie rozciągniętych odstępów).
%
\begin{center}
	\LARGE{\textbf{Feria sexta in Passione et Morte Domini}}\\ \smallskip
	\small{Wielki Piątek Męki i Śmierci Pańskiej\\ \smallskip Stacja u Św. Św. Krzyża Jerozolimskiego}
\end{center} \vspace{1cm}

	\textbf{Wprowadzenie} Przeżywamy dziś w liturgii Wielki Piątek Męki i Śmierci Pańskiej. Od niepamiętnych czasów ten dzień jest dla chrześcijan dniem postu i żałoby.	
\par W rocznicę krwawej ofiary Chrystusa na Krzyżu, Kościół wstrzymuje się od odprawiania ofiary bez\-krwa\-wej. W porze popołudniowej odprawia się specjalne nabożeństwo, którego forma sięga starożytności chrześcijańskiej.
\par Dzisiejsza liturgia składa się z czterech części. Pierwszą z nich jest Liturgia słowa, w której usłyszymy opis Męki Pańskiej według Św. Jana, oraz w uroczystej Modlitwie powszechnej zaniesiemy modlitwy za całą ludzkość odkupioną Krwią Chrystusa. Kulminacyjnym punktem liturgii jest adoracja Krzyża. Obrzęd ten wziął swój początek od adoracji relikwii Świętego Krzyża, która odbywała się w Jerozolimie już w IV wieku. Trzecią częścią dzisiejszej liturgii jest Komunia Święta. Choć dzisiaj Msza Święta nie jest sprawowana, Kościół pragnie zaznaczyć jedność ofiary Wieczernika z krwawą Ofiarą Pana Jezusa na Krzyżu. W ostatniej części nastąpi przeniesienie Najświętszego Sakramentu do Bożego Grobu, który uzmysławia wielkość ofiary Zbawiciela i ciężkość grzechu, który wymagał takiego okupu.
\par Za chwilę kapłan w otoczeniu asysty uda się przed ołtarz i położy się krzyżem, w geście najgłębszej adoracji, na znak śmierci Chrystusa. Wszyscy pozostajemy w tym czasie w postawie klęczącej.
\par W modlitewnym skupieniu oczekujmy teraz na rozpoczęcie liturgii. \bigskip
	
	\textbf{Liturgia Słowa} Liturgia Słowa ukazuje nam tajemnicę śmierci Jezusa Chrystusa. Cierpiący Sługa Boży, doświadczany we wszystkim na nasze podobieństwo, z wyjątkiem grzechu, jest posłuszny Ojcu aż do końca.
\par W Pasji św. Jana, świadka śmierci Jezusa, na słowa \emph{,,I skłoniwszy głowę, wyzionął ducha''}, wszyscy klękniemy na znak, że śmierć Chrystusa uobecnia się tu i teraz dla nas. \bigskip
	
	\textbf{Modlitwa powszechna} Śmierć Chrystusa na Krzyżu ma charakter uniwersalny, dlatego też Kościół modli się zawsze za cały świat i w tym dniu wyjątkowo to podkreśla. Dzisiejsza modlitwa wiernych ma szczególnie uroczysty charakter -- zachowała ona pierwotną formę zebrania liturgicznego gminy rzymskiej. W tym roku biskup ordynariusz wyznaczył dodatkowe wezwania w związku z panującą sytuacją epidemii oraz za ofiary katastrofy lotniczej w Smoleńsku. Każda z dwunastu modlitw składa się z wezwania zapowiadającego intencję, chwili ciszy i prośby kapłana, na którą wierni odpowiadają: \emph{Amen}.\bigskip
	
	\textbf{Adoracja Krzyża} Punktem kulminacyjnym dzisiejszej liturgii jest Adoracja Krzyża. Krzyż, na którym \emph{,,zawisło zbawienie świata''} zostanie wniesiony do świątyni i odsłonięty w trzech etapach. Na każde wezwanie kapłana uklękniemy, odpowiadając: \emph{Pójdźmy z pokłonem}.
\par Porządek adoracji będzie wyglądał następująco:\\
Cześć Krzyżowi poprzez przyklęknięcie, bez bezpośredniego kontaktu oddadzą wspólnie wszyscy wierni. Dzisiejsza Adoracja Krzyża pozwala uzyskać odpust zupełny, pod zwykłymi warunkami, tj.: stan łaski uświęcającej, przyjęcie Komunii świętej, brak jakiegokolwiek przywiązania do grzechu, nawet powszedniego oraz odmówienie modlitwy w intencjach wyznaczonych przez Ojca Świętego. \newpage % \bigskip
	
	\textbf{Komunia Święta (podczas przygotowania ołtarza)} Rozpoczyna się trzecia część wielkopiątkowej liturgii -- Komunia święta. Kościół, chociaż nie sprawuje dziś Mszy Świętej, pragnie karmić nas Ciałem Chrystusa.
	\par Wiernych, którzy łączą się z nami poprzez transmisję internetową i nie mogą przystąpić do Komunii świętej, pouczamy o możliwości skorzystania z praktyki Komunii duchowej.
	\par Osoby, które nie mają możliwości przystąpienia do Spowiedzi sakramentalnej, zachęcamy do wzbudzenia żalu doskonałego, to jest żalu płynącego z miłości nade wszystko ku Bogu. Taki żal połączony z mocnym postanowieniem przystąpienia do Spowiedzi sakramentalnej, gdy tylko będzie to możliwe, przynosi przebaczenie grzechów śmiertelnych, grzechy te trzeba jednak wyznać na najbliższej Spowiedzi.
\par Ze szczerą wdzięcznością przystąpmy teraz do stołu Pańskiego. \bigskip

	\textbf{Przed procesją do Grobu Pańskiego} Dobiega końca dzisiejsza liturgia. Za chwilę, na podobieństwo Józefa z Arymatei i Nikodema, przeniesiemy Najświętsze Ciało Chrystusa do Bożego Grobu. Aż do poranka wielkanocnego adorować będziemy Zbawiciela, dziękując Mu za owoce Jego męki i śmierci. \bigskip
	
	Jutrzejsza liturgia Wigilii Paschalnej w naszej kaplicy rozpocznie się o godzinie 20:30.
	
	Wszystkie celebracje Triduum Paschalnego są transmitowane na stronie naszej parafii, na Facebooku: \href{https://www.facebook.com/Parafia-\%C5\%9Aw-Papie\%C5\%BCa-Jana-Paw\%C5\%82a-II-na-Osiedlu-Dobrzec-w-Kaliszu-104456801212546/}{Parafia Św. Papieża Jana Pawła II na Osiedlu Dobrzec w Kaliszu}.
\end{document}
