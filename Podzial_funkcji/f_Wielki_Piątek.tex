\documentclass[10pt,oneside,final,notitlepage,a4paper,wide]{mwart}
\usepackage[utf8]{inputenc} 
\usepackage{polski} 
\usepackage{graphicx}
\usepackage{setspace}
\usepackage{amsfonts}

\onehalfspacing
\setlength{\parindent}{1cm}
\renewcommand{\labelitemi}{$\bullet$}

\usepackage[]{hyperref}
\usepackage{hyperref,xcolor}% http://ctan.org/pkg/{hyperref,xcolor}
\usepackage{makeidx}

\definecolor{orangelink}{rgb}{0.7,0.18,0.1}

\hypersetup{
    pdftitle={Feria sexta in Passione et Morte Domini},
    pdfdisplaydoctitle=true,
    pdfauthor={JanekR_Prorok},
    pdfsubject={Source: https://github.com/JanekRProrok/Triduum_Sacrum},
    pdfcreator={Texmaker, MiKTeX},
    pdfproducer={},
    pdfinfo={},
    pdfkeywords={},
    bookmarks=true,
    bookmarksnumbered=true,
    bookmarksopen=true,
    bookmarksopenlevel=1,
    pdfpagelabels=true,
    pdfpagemode=UseOutlines,
    unicode=true,
    pdftoolbar=true,
    pdfmenubar=true,
    pdffitwindow=false,
    pdfstartview=Fit,
    pdfnewwindow=true,
    colorlinks=true,
    linkcolor=orangelink,
    citecolor=orangelink,
    filecolor=orangelink,
    urlcolor=orangelink,
    pdfpagelayout=TwoPageRight,
}
% http://www.tug.org/applications/hyperref/manual.html#x1-120003.8

%------------------------------------------------------------------------------%
%------------------------------------------------------------------------------%

	% Wypełnienie ... \dotline{długość+0,1cm}.
\def\dotfill#1{\cleaders\hbox to #1{.}\hfill}
\newcommand\dotline[2][0,1cm]{\leavevmode\hbox to #2{\dotfill{#1}\hfil}}	
	
%------------------------------------------------------------------------------%
	
\begin{document}
% \sloppy - Wymuszenie nieprofesjonalnego trzymania się w marginesach (kosztem dziwnie rozciągniętych odstępów).
%
\begin{center}
	\LARGE{\textbf{Feria sexta in Passione et Morte Domini}}\\ \smallskip
	\small{Wielki Piątek Męki i Śmierci Pańskiej\\ \smallskip Stacja u Św. Św. Krzyża Jerozolimskiego}
\end{center} \vspace{1cm}

\begin{enumerate}
	\item Kolejność procesji wejścia -- według wzrostu:
	\begin{enumerate}
		\item Precentor --
			\begin{itemize}
				\item Welon naramienny
			\end{itemize}
		\item Nawikulariusz -- 
		\item Turyferariusz --
			\begin{itemize}
				\item Okadzenia przy przeniesieniu Najświętszego Sakramentu do Grobu Pańskiego
			\end{itemize}
		\item Krucyferariusz --
			\begin{itemize}
				\item Ukazanie Krzyża
			\end{itemize}
		\item Akolici --
			\begin{itemize}
				\item Ukazanie Krzyża
			\end{itemize}
		\item Ministranci
		\item Ministranci ołtarza --
			\begin{itemize}
				\item Ubranie ołtarza
				\item Obnażenie ołtarza
			\end{itemize}
		\item Lektorzy -- 
		\item Klerycy
		\item Kapłani
		\item Celebrans -- 
		\item Ceremoniarz -- 
	\end{enumerate}
\smallskip
	\item Funkcje niezależne:
	\begin{enumerate}
		\item Komentator --
		\item Kołatki -- 
		\item Modlitwa powszechna -- 
		\item Pateny --
		\item Ministranci do pomocy przy adoracji Krzyża po Mszy --
	\end{enumerate}
\medskip
	\item Męka Pańska według św. Jana:
	\begin{itemize}
		\item[$\maltese$] Chrystus -- Celebrans/Diakon
		\item[\textbf{E.}] Ewangelista -- 
		\item[\textbf{T.}] Klika osób lub Tłum -- 
		\item[\textbf{I.}] Osoby pojedyncze -- 
	\end{itemize}
\end{enumerate}
\end{document}
